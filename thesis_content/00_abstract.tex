% 要約
\begin{abstract}
本研究は、マインドアップロード(全脳エミュレーション)を長期目標とし、VR×EEG研究が果たす役割を5年計画(Phase 1-3)から30-50年スパン(Phase 3.5-6)までの段階的ロードマップとして体系化した。計測(Measure)・理解(Understand)・制御(Control)のMECE軸で整理し、高密度EEG(≥128ch)と個人MRIを用いたソース推定、自己教師あり表現学習によるデコーディング、tES/TMSによる閉ループ刺激、在宅EEGコホートによる社会実装を統合した。EEGの空間分解能限界(浅部15mm/中深度25mm、深部は融合必須)、脳梁BMIの電極密度・長期安定性・軸索再生の未解決点、記憶転送の神経科学的未解明点を明示し、現実的なKPIと中間マイルストーン(Phase 3.5の新設)を提示した。これにより、科学的誠実性を担保しつつ、非侵襲中心のEEG研究から侵襲的BMI・大規模シミュレーションへの橋渡し条件を定量化した。
\end{abstract}


